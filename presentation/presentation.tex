\documentclass{beamer}
\usetheme{Madrid}
\usecolortheme{default}
\useoutertheme{split}
\useinnertheme{circles}
\usepackage{xcolor}
\usepackage{mathrsfs}
\usepackage{amsmath}
\usepackage{amssymb}
\usepackage{bm}
\usepackage{booktabs}
\usepackage{hyperref}
\usepackage{animate}
\usepackage{caption}
\captionsetup[figure]{font=tiny}

\usepackage[
backend=biber,
style=alphabetic,
sorting=ynt 
]{biblatex}
\addbibresource{bibliography.bib}

% --- Couleurs CS ---
\definecolor{CSred}{RGB}{160,32,60}
\definecolor{CSgrey}{RGB}{136,121,150}
\definecolor{UPSred}{RGB}{97,21,58}
\setbeamercolor{titlelike}{bg=CSred}
\setbeamerfont{title}{series=\bfseries}
\setbeamercolor{palette primary}{bg=CSred,fg=white}
\setbeamercolor{palette secondary}{bg=CSred,fg=white}
\setbeamercolor{palette tertiary}{bg=CSgrey,fg=white}
\setbeamercolor{palette quaternary}{bg=CSgrey,fg=white}
\setbeamercolor{structure}{fg=CSgrey}

\title[IPI]{Inverse Problems and Imaging}
\subtitle{Presentation - Project Assignment}
\author[Jamal, Martini]{Adonis~JAMAL \quad Jean-Vincent~MARTINI\\Reverse-time and Kirchhoff migration}
\institute[ENS]{Ecole Normale Supérieure Paris-Saclay\\MVA 2025-2026}
\date[March 27th 2026]{March 27th, 2026}

\titlegraphic{
\centering
\includegraphics[height=1cm]{img/logo_ens-ps.png}
\hspace{1cm}
\includegraphics[height=1.2cm]{img/logo_mva.jpg}
\hspace{1cm}
\includegraphics[height=1cm]{img/Logo_CentraleSupelec.svg.png}
}

\begin{document}

\frame{\titlepage}


% ==============================================================================
% SECTION 1: INTRODUCTION
% ==============================================================================

\begin{frame}{Overview and Objectives}
\begin{columns}
\begin{column}{0.55\textwidth}
\textbf{Goal:} Reconstruct the location of a point-like reflector from time-harmonic or broadband measurement data in 2D, using synthetic array data.

\bigskip
\textbf{Configurations studied:}
\begin{itemize}
  \item \textbf{Full aperture}: circular array of $N$ transducers
  \item \textbf{Partial aperture}: linear array along the $x$-axis
  \item \textbf{Time-dependent}: broadband signal emission
  \item \textbf{Noisy setting}: additive complex Gaussian noise
\end{itemize}
\end{column}
\begin{column}{0.42\textwidth}
\textbf{Methods compared:} Reverse-Time (RT) migration, Kirchhoff Migration (KM), MUSIC-type. 

\bigskip
\textbf{Workflow:}
\begin{enumerate}
  \item Derive the Green's function and generate synthetic data via Born approximation
  \item Define and implement RT, KM, MUSIC functionals
  \item Analyze results
  \item Study stability under noise (Monte Carlo)
\end{enumerate}
\end{column} 
\end{columns}
\end{frame}


% ==============================================================================
% SECTION 2: THEORETICAL TOOLS
% ==============================================================================

\begin{frame}{Green's Function and Data Model}
\textbf{Green's function.} We assume propagation speed $c_0 = 1$. The 2D homogeneous Green's function $\hat{G}_0(\omega, x, y)$ satisfies:
$$\Delta_x \hat{G}_0 + \omega^2 \hat{G}_0 = -\delta(x-y), \quad x \in \mathbb{R}^2$$
with the Sommerfeld radiation condition. It is given explicitly by:
$$\hat{G}_0(\omega, x, y) = \frac{i}{4} H_0^{(1)}(\omega |x-y|)$$
where $H_0^{(1)}(s) = J_0(s) + iY_0(s)$ is the Hankel function of the first kind.

\bigskip
\textbf{Data model (Born approximation).} For a point-like reflector at $\bm{x}_{\mathrm{ref}}$, the data matrix entry for receiver $r$ and source $s$ is:
$$\hat{u}_{rs}(\omega) = \omega^2 \hat{G}_0(\omega, \bm{x}_r, \bm{x}_{\mathrm{ref}})\, \hat{G}_0(\omega, \bm{x}_{\mathrm{ref}}, \bm{x}_s)$$
which can be written in matrix form as:
$$\hat{U}(\omega) = -\frac{\omega^2}{16}\, \mathbf{v}\mathbf{v}^T, \quad v_t = H_0^{(1)}(\omega|\bm{x}_t - \bm{x}_{\mathrm{ref}}|)$$
\end{frame}

\begin{frame}{RT, KM and MUSIC Imaging Functionals}
\textbf{Reverse-Time (RT) migration:}
$$\mathcal{I}_{RT}(\bm{x}) = \frac{1}{2\pi}\int_{-\infty}^{\infty} d\omega \sum_{r,s} \overline{\hat{G}_0(\omega, \bm{x}, \bm{x}_r)}\,\hat{u}_{rs}(\omega)\,\overline{\hat{G}_0(\omega, \bm{x}_s, \bm{x})}$$

\textbf{Kirchhoff Migration (KM):} from RT, with $\hat{G}_0(\omega,\bm{x},\bm{y})\approx e^{i\omega\mathcal{T}(\bm{x},\bm{y})}$:
$$\mathcal{I}_{KM}(\bm{x}) = \frac{1}{2\pi}\int_{-\infty}^{\infty} d\omega \sum_{r,s} e^{-i\omega|\bm{x}-\bm{x}_r|}\,\hat{u}_{rs}(\omega)\,e^{-i\omega|\bm{x}_s - \bm{x}|}$$

\textbf{MUSIC} (partial aperture, noise-subspace variant):
$$\mathcal{I}_{MU}(\bm{x}) = \frac{1}{1 - |\langle \hat{g}(\omega, \bm{x}),\, v_1\rangle|^2}$$
where $\hat{g}(\omega,\bm{x}) = (\hat{G}_0(\omega, \bm{x}, x_r))_{r=1,\dots,N}$ and $v_1$ is the first singular vector of $\hat{U}$.

\textbf{Theoretical focal spots:} Full aperture $\sim J_0^2(\omega|x-x_{\mathrm{ref}}|)$; Partial aperture: $\operatorname{sinc}^2$ in cross-range, Fresnel integral in range.
\end{frame}


% ==============================================================================
% SECTION 3: RESULTS - FULL APERTURE
% ==============================================================================

\begin{frame}{Full Aperture -- RT, KM and Theoretical Focal Spot}
\textbf{Setup:} $N=100$ transducers on a circle, $R_0=100$, $\omega=2\pi$, $\bm{x}_{\mathrm{ref}}=(10,20)$.

\begin{columns}
\begin{column}{0.58\textwidth}
\includegraphics[width=\textwidth]{../figures/Full_Aperture.png}
\end{column}
\begin{column}{0.40\textwidth}
\small
\begin{itemize}
  \item RT and KM images are very similar; minor discrepancies come from the approximation used in KM.
  \item The theoretical $J_0^2$ focal spot differs noticeably: it is derived under the asymptotic limit $R_0\to\infty$, $N\to\infty$.
  \item Near the reflector, all three profiles agree qualitatively.
\end{itemize}
\end{column}
\end{columns}
\end{frame}

\begin{frame}{Full Aperture -- Multiple Reflector Positions}
\textbf{Effect of reflector position:} reflector moved to $(0,0)$, $(0,100)$, $(100,0)$.

\begin{center}
\includegraphics[width=0.82\textwidth]{../figures/Full_Aperture_Multiple_Reflectors.png}
\end{center}

\small As the reflector approaches the array boundary, the asymptotic assumptions break down and the discrepancy between RT/KM and the theoretical $J_0^2$ profile becomes more pronounced.
\end{frame}

\begin{frame}{Full Aperture -- Far-Field Regime}
\textbf{Far-field regime:} $R_0 = 1000$, $N = 2000$.

\begin{center}
\includegraphics[width=0.82\textwidth]{../figures/Full_Aperture_Far_Field.png}
\end{center}

\small With a much larger array and many more transducers, the RT and KM functionals converge closely to the theoretical $J_0^2$ focal spot, confirming the validity of the asymptotic formula in the appropriate regime.
\end{frame}


% ==============================================================================
% SECTION 4: RESULTS - PARTIAL APERTURE
% ==============================================================================

\begin{frame}{Partial Aperture -- RT, KM and MUSIC}
\textbf{Setup:} $N=100$ transducers on a linear array, $R_0=100$, $\omega=2\pi$, $\bm{x}_{\mathrm{ref}}=(10,20)$.

\begin{columns}
\begin{column}{0.56\textwidth}
\includegraphics[width=\textwidth]{../figures/Partial_Aperture.png}
\end{column}
\begin{column}{0.42\textwidth}
\small
\begin{itemize}
  \item RT, KM and MUSIC closely resemble each other.
  \item \textbf{Cross-range ($x$):} RT reproduces the main peak and secondary lobes almost perfectly.
  \item \textbf{Range ($z$):} a mirror image appears at $-z_{\mathrm{ref}}$ due to the linear geometry (transducers on the $x$-axis cannot distinguish $z$ from $-z$).
\end{itemize}
\end{column}
\end{columns}
\end{frame}

\begin{frame}{Partial Aperture -- Influence of Array Length $R_0$}
\begin{columns}
\begin{column}{0.56\textwidth}
\begin{center}
\includegraphics[width=\textwidth]{../figures/Partial_Aperture_Multiple_R0.png}
\end{center}
\end{column}

\begin{column}{0.42\textwidth}
\textbf{Varying $R_0$:} smaller aperture $\Rightarrow$ coarser cross-range resolution (larger $r_c = \lambda|x_{\mathrm{ref}}|/R_0$).

\bigskip
\small As $R_0$ decreases, the focal spot widens in the cross-range direction, consistent with the theoretical $\operatorname{sinc}^2$ prediction. 
\end{column}
\end{columns}
\end{frame}

\begin{frame}{Partial Aperture -- Multiple Reflector Positions}
\begin{columns}
\begin{column}{0.56\textwidth}
\begin{center}
\includegraphics[width=\textwidth]{../figures/Partial_Aperture_Multiple_Reflectors.png}
\end{center}
\end{column}
\begin{column}{0.42\textwidth}
\textbf{Varying $\bm{x}_{\mathrm{ref}}$:} moving the reflector further from the array increases the range and degrades cross-range resolution.

\bigskip
\small The focal spot characteristics scale with the stand-off distance $|x_{\mathrm{ref}}|$, in agreement with the theoretical expressions for $r_c$ and $r_l$.
\end{column}
\end{columns}
\end{frame}


% ==============================================================================
% SECTION 5: RESULTS - TIME-DEPENDENT
% ==============================================================================

\begin{frame}{Time-Dependent (Broadband) -- Partial Aperture}
\textbf{Setup:} $N=40$, $R_0=20$, broadband signal $\hat{f}(\omega) = \mathbf{1}_{[\omega_0-B,\,\omega_0+B]}(\omega)$, $\omega_0=2\pi$, $B=0.05\omega_0$, $\bm{x}_{\mathrm{ref}}=(0,100)$.

\begin{columns}
\begin{column}{0.56\textwidth}
\includegraphics[width=\textwidth]{../figures/Time_Dependent.png}
\end{column}
\begin{column}{0.42\textwidth}
\small
\begin{itemize}
  \item The focal spot is significantly more localized than in the monochromatic case: broader frequency content improves coherent focusing.
  \item Cross-range: RT matches the $\operatorname{sinc}^2$ main peak and secondary lobes well, confirming the theoretical prediction.
  \item Range: the profile is consistent with the theoretical $|\operatorname{sinc}(2B|z-z_{\mathrm{ref}}|)|$ prediction.
\end{itemize}
\end{column}
\end{columns}
\end{frame}

\begin{frame}{Time-Dependent -- Influence of Bandwidth $B$}
\begin{columns}
\begin{column}{0.56\textwidth}
\begin{center}
\includegraphics[width=0.7\textwidth]{../figures/Time_Dependent_Multiple_Bandwidths.png}
\end{center}
\end{column}
\begin{column}{0.42\textwidth}
\textbf{Varying bandwidth $B$:} wider bandwidth $\Rightarrow$ sharper range resolution ($r_l \propto 1/B$), cross-range resolution unchanged.

\bigskip
\small Increasing $B$ dramatically reduces the range sidelobe level and sharpens the range profile, confirming the $\operatorname{sinc}$ dependence on bandwidth.
\end{column}
\end{columns}
\end{frame}


% ==============================================================================
% SECTION 6: STABILITY / NOISE
% ==============================================================================

\begin{frame}{Stability -- Noisy Model}
\textbf{Noise model.} The recorded signals are perturbed by complex Gaussian noise:
$$\hat{u}_{rs}^{\mathrm{noisy}}(\omega) = \hat{u}_{rs}(\omega) + W_{rs}^{(1)}(\omega) + i\, W_{rs}^{(2)}(\omega)$$
where $W_{rs}^{(1)}, W_{rs}^{(2)} \sim \mathcal{N}(0,\, \sigma^2/2)$ i.i.d.

\bigskip
\textbf{Configurations revisited:}
\begin{itemize}
  \item Full aperture (RT \& KM)
  \item Partial aperture (RT, KM \& noise-subspace MUSIC)
  \item Time-dependent broadband (RT \& KM)
\end{itemize}

\bigskip
\textbf{Analysis:} For each configuration, we plot images at increasing $\sigma$, then quantify localization error via Monte Carlo simulation (multiple noise realizations).

\bigskip
\textit{Note on MUSIC:} we use the noise-subspace variant $\mathcal{I}_{MU}(\bm{x}) = 1/(1-|\langle\hat{g},v_1\rangle|^2)$ to expose its super-resolution behaviour and its sensitivity to noise.
\end{frame}

\begin{frame}{Stability -- Full Aperture under Noise}

\begin{center}
\includegraphics[width=0.65\textwidth]{../figures/Noise_Full_Aperture.png}
\end{center}

\small RT and KM both degrade gracefully as $\sigma$ increases. Coherent integration over many transducer pairs provides a natural averaging effect that partially mitigates noise.
\end{frame}

\begin{frame}{Stability -- Partial Aperture under Noise}

\begin{center}
\includegraphics[width=0.7\textwidth]{../figures/Noise_Partial_Aperture.png}
\end{center}

\small RT and KM remain robust for moderate noise levels. The noise-subspace MUSIC functional is sharper at low noise but degrades more abruptly: once the noise level is sufficient to perturb the dominant singular vector $v_1$, the localization quality deteriorates rapidly.
\end{frame}

\begin{frame}{Stability -- Time-Dependent Broadband under Noise}

\begin{center}
\includegraphics[width=0.88\textwidth]{../figures/Noise_Time_Dependent.png}
\end{center}

\small The broadband RT functional retains a well-defined peak even at elevated noise levels, benefiting from frequency diversity in addition to spatial averaging.
\end{frame}

\begin{frame}{Stability -- Monte Carlo Analysis: RT vs.\ MUSIC}
\textbf{RMSE of localization vs.\ noise level $\sigma$} (multiple realizations, partial aperture).

\bigskip
\begin{itemize}
  \item \textbf{RT:} localization error grows slowly and steadily with $\sigma$. Coherent back-propagation integrates over all source--receiver pairs, averaging out random noise contributions.
  \item \textbf{MUSIC:} error remains very low for small $\sigma$ (super-resolution), then rises sharply beyond a threshold. Once noise corrupts the dominant singular vector $v_1$, the functional fails abruptly.
\end{itemize}

\begin{columns}
\begin{column}{0.5\textwidth}
\includegraphics[width=\textwidth]{../figures/Stability_vs_Noise.png}
\end{column}
\begin{column}{0.48\textwidth}

\bigskip
$\Rightarrow$ RT is more robust; MUSIC offers higher precision at low noise but is fragile.
\end{column}
\end{columns}
\end{frame}


% ==============================================================================
% SECTION: CONCLUSION
% ==============================================================================

\begin{frame}{Conclusion}
\begin{columns}
\begin{column}{0.55\textwidth}
\textbf{Summary of findings:}
\small
\begin{itemize}
  \item \textbf{Full aperture:} RT and KM agree closely; both approach the theoretical $J_0^2$ focal spot as $R_0,N\to\infty$.
  \item \textbf{Partial aperture:} linear geometry introduces a mirror artifact in range; cross-range resolution scales as $r_c=\lambda|x_{\mathrm{ref}}|/R_0$. MUSIC provides a sharper image.
  \item \textbf{Broadband:} increased bandwidth drastically improves range resolution ($r_l \propto 1/B$), consistent with $\operatorname{sinc}$ theory.
  \item \textbf{Noise:} RT migration degrades gracefully; MUSIC is precision-first but collapses beyond a noise threshold.
\end{itemize}
\end{column}
\begin{column}{0.43\textwidth}
\textbf{Perspectives:}
\small
\begin{itemize}
  \item Extension to heterogeneous or anisotropic media.
  \item Advanced regularization to stabilize MUSIC at high noise.
  \item Investigation of multiple reflectors and mutual interference.
  \item Time-reversal approaches in random media.
\end{itemize}

\bigskip
\textbf{Key take-away:} Array geometry, bandwidth and noise level are the three dominant factors controlling resolution and stability of RT/KM imaging functionals.
\end{column}
\end{columns}
\end{frame}


% ==============================================================================
% SECTION : BIBLIOGRAPHY
% ==============================================================================
\begin{frame}[allowframebreaks]{Bibliography}
\frametitle{Bibliography}
Ammari et al., \textit{Mathematical and Statistical Methods for Multistatic Imaging}, Springer, 2013.
\end{frame}


\end{document}